%%%%%%%%%%%%%%%%%%%%%%%%%%%%%%%%%%%%%%%%%
% cv-friggeri-x 1.0 (01/01/2016)
% XeLaTeX Template
%
% Based on:
% Friggeri Resume/CV
% Version 1.2 (3/5/15)
%
% Original author:
% Adrien Friggeri (adrien@friggeri.net)
% https://github.com/afriggeri/CV
%
% Modified by:
% Nadorrano
% https://github.com/Nadorrano/cv-friggeri-x
%
% License:
% MIT (https://opensource.org/licenses/MIT)
% CC BY-NC-SA 3.0 (http://creativecommons.org/licenses/by-nc-sa/3.0/)
%
% Important notes:
% This template needs to be compiled with XeLaTeX and the bibliography,
% if used, needs to be compiled with biber rather than bibtex.
%
%%%%%%%%%%%%%%%%%%%%%%%%%%%%%%%%%%%%%%%%%

\documentclass[a4paper]{cv-friggeri-x}
% Add `a4paper` to set a4 paper size
% Add `nocolors` to remove colors from the document
% Add `lightheader` to change the dark background of the header to white

\usepackage{marvosym} % needed to print glyphs for email, cell phone etc.
\usepackage{soul} % used on the quote
\usepackage[none]{hyphenat}
\usepackage{afterpage}
\setstcolor{lightgray}

\addbibresource{bibliography.bib} % Specify the bibliography file to include publications

\begin{document}

\header{Daniel}{ Kneipp}{Senior Platform Engineer}%{''Do what you can, with what you have, where you are.'' -- Theodore Roosevelt} % Your name and current job title/field

%----------------------------------------------------------------------------------------
%	SIDEBAR SECTION
%----------------------------------------------------------------------------------------

\begin{aside} % In the aside, each new line forces a line break
\section{Contact}
% \pin \hfill 119 Gloriosa St,
% Belo Horizonte, MG
% 305190-490, Brazil
% ~
% {\Large\textcolor{gray}{\Telefon}} \hfill +0 (000) 111 1111
{\Large\textcolor{gray}{\Mobilefone}} \hfill +31 6 16 11 35 47
{\Large\textcolor{gray}{\Letter}} \hfill \href{mailto:daniel.kneipp@outlook.com}{daniel.kneipp@\\outlook.com}
~
% \flogo \hfill \href{http://facebook.com/johnsmith}{fb://jsmith}
% \tlogo \hfill \href{http://twitter.com/johnsmith}{@jsmith}
\llogo \hfill \href{https://www.linkedin.com/in/daniel-kneipp/}{in://daniel-kneipp}
\githublogo \hfill \href{https://github.com/DanielKneipp}{github://DanielKneipp}
\gitlablogo \hfill \href{https://gitlab.com/DanielKneipp}{gitlab://DanielKneipp}
% \vklogo \hfill \href{http://vk.com/johnsmith}{vk://johnsmith}
% ~
% \href{http://www.smith.com}{http://www.smith.com}
\section{Languages}
Brazilian Portuguese \hspace{5mm}\null
\textit{\footnotesize{[Mother tongue]}}
English  \hspace{5mm}\null
\textit{\footnotesize{[Professional working proficiency]}}
\section{Programming}
Python, Bash,
C++, Go, R
\section{Skills}
Cloud Infrastructure: \hspace{5mm}\null
\grade{4}
DevOps:               \hspace{5mm}\null
\grade{4}
Machine Learning:     \hspace{5mm}\null
\grade{3}
Computer Vision:      \hspace{5mm}\null
\grade{3}
Back-end Dev:         \hspace{5mm}\null
\grade{2.5}
\end{aside}

%----------------------------------------------------------------------------------------
%	ABOUT ME SECTION
%----------------------------------------------------------------------------------------

\section{About me}
I'm a Platform Engineer with over 10 years of professional experience working with Computer Vision, Machine Learning and Cloud Infrastructure. I also have a Master's degree in Computer Science (focus on DNA Computing), aggregating theoretical knowledge to the practical experience.

Currently I'm working as a Senior Platform Engineer, maintaining a highly available Kubernetes environment and supporting teams of developers on their operations.
% I am objective and focused on results. With entrepreneurship in my veins, I've spent many years studying and learning, from Convolutional Neural Networks to SWOT Analysis and Nash Equilibrium, trying to make a difference where I go. I know that it looks like a Miss Universe candidate speech asking for world peace (joke), but I really work hard to make some impact. Please, checkout my code repository (\href{https://github.com/DanielKneipp}{https://github.com/DanielKneipp}) to see some cool stuff.


%----------------------------------------------------------------------------------------
%	WORK EXPERIENCE SECTION
%----------------------------------------------------------------------------------------

\section{Experience}

% \subsection{Full time}

\begin{entrylist}

%------------------------------------------------

\entry
    {2022--Now}
    {Senior Platform Engineer}
    {Bitvavo -- Netherlands}
    {\begin{itemize}
        \item Implemented solution to access private corporate resource, enabling the company to follow Zero Trust principles with IdP metadata for authorization and monitoring of internal network activity;
        \item Helped maintain several integrated Kubernetes clusters on a multi AWS account setup with Istio for networking and Grafana for monitoring, all managed via terraform following the GitOps approach;
        \item Supported multiple teams on maintaining services managed via ArgoCD, as well as conducting zero downtime upgrades of highly business critical applications;
        \ Also helped maintain legacy infrastructure running on long-standing EC2 instances;
        \item Designed solution on AWS for a new infrastructure focused on low latency using bare-metal co-located EC2 instances leveraging placements groups.
    \end{itemize}}

\entry
    {2021--2022}
    {Senior System Engineer}
    {Backbase -- Netherlands}
    {\begin{itemize}
        \item Developed and supported automation pipelines with Jenkins and Github Actions;
        \item Conducted workshops and knowledge sharing sessions regarding DevOps culture and tooling to support developers across several teams (focus on local development/testing for cloud-native solutions);
        \item Provisioned SaaS environments with restoration mechanisms to the compute and storage layers to be served as a reference working API for the whole company (trainees and developers);
        \item Supported pipelines of projects in Java, Javascript, Typescript and Go for multiple teams with automated interactions with Helm, Terraform, Jfrog, Kubernetes and custom internal tooling.
    \end{itemize}}

\entry
    {2019--2021}
    {Site Reliability Engineer}
    {MOST Specialist Technologies -- Brazil}
    {\begin{itemize}
        \item Designed automated packing and testing processes of containerized services with Gitlab pipelines, improving update rate from weakly to daily;
        \item Blue-Green deployments and Rolling releases with AWS EC2, Fargate, ECS and CloudFormation;
        \item Deployed and maintained a monitoring system and request tracing with Elastic Stack (Elasticsearch and Kibana), reducing the time of incident response from hours to minutes;
        \item Infrastructure automation using Ansible to configure ephemeral development instances, Packer for automated AWS AMIs creation and Terraform for the infrastructure provisioning.
    \end{itemize}}

\entry
    {2017-2018}
    {Machine Learning Engineer}
    {MOST Specialist Technologies -- Brazil}
    {\begin{itemize}
        \item Developed ML algorithms for document detection in images and ID recognition (demo available at \href{mostqi.com}{mostqi.com});
        \item Developed back-end services in Python and Go to serve real-time inferences;
        \item Worked with document classification and clustering using its textual content;
        \item Used Data Engineering and Data Visualization concepts for data preparation, cleaning and labelling unification;
        \item Developed tools for training management and performance monitoring regarding accuracy and hyper-parametrization.
    \end{itemize}}

%------------------------------------------------

% \end{entrylist}

% \newpage

% \subsection{Part time}

% \begin{entrylist}

\entry
    {2016--2017}
    {Computer Vision Researcher}% RHAE SPECIAL PROGRAM
    {Invent Vision -- Brazil}
    {Deep Learning research for Computer Vision applications. Implemented a set of tools to speedup the development (e.g. synthetic dataset generation, dataset management, etc.) and deployed image classifiers in embedded systems (NVIDIA Jetson).}

\entry
    {2015--2016}
    {Computer Vision Intern}
    {Invent Vision -- Brazil}
    {Research and provisioning of distributed computing systems (based on Hadoop and Spark), developing simple applications made to run across clusters.}

\entry
    {2013--2014}
    {Undergraduate Researcher}
    {Invent Vision -- Brazil}
    {Development of an efficient drowsiness detector based on face expressions (using face and eye tracking algorithms). Deployment made on x86 computers and ARM embedded systems.}


%------------------------------------------------

\end{entrylist}

% \newpage

%----------------------------------------------------------------------------------------
%	EDUCATION SECTION
%----------------------------------------------------------------------------------------

%\afterpage{%
%\newgeometry{left=2.0cm,top=1.0cm,right=2.0cm,bottom=1.0cm,nohead,nofoot}

\section{Education}

\begin{entrylist}

\entry
    {2016--2018}
    {Master {\normalfont of Science}}
    {Federal University of Minas Gerais (UFMG)}
    {\emph{Comptuer Science} --
    \footnotesize{NanoComp lab. member (\href{http://www.nanocomp.dcc.ufmg.br/}{http://www.nanocomp.dcc.ufmg.br/}).}\\
    \normalsize{My research area was DNA Computing. The objective was to propose functional chemical circuits for classification tasks using Chemical Reaction Networks theory as a programming language and DNA strands as the hardware. One of the results of my research is a R package to simulate logic circuits based on DNA. See \href{https://github.com/DanielKneipp/DNAr}{https://github.com/DanielKneipp/DNAr} to know more.}}

\entry
    {2012--2015}
    {Bachelor {\normalfont of Science}}
    {Federal University of Vi\c cosa (UFV)}
    {\emph{Computer Science}\\
    I received the Presidente Bernardes Medal for my academic excellence. In my undergraduate thesis I developed an algorithm based on a bio-inspired meta-heuristic to solve a combinatorial optimization problem.}


% \entry
%     {2010--2011}
%     {Technician's {\normalfont Degree}}
%     {SENAI School}
%     {\emph{Informatics}\\
%     I Studied the basics of Computer Architecture, Software Development and Network Infrastructure.}

% \entry
%     {}
%     {\\Independent {\normalfont Courses}}
%     {}
%     {\begin{itemize}%
%         \item Deep Learning -- \textit{Google} | \textit{Udacity}%
%         \item Machine Learning -- \textit{Stanford University} | \textit{Coursera}%
%     \end{itemize}}

%------------------------------------------------

\end{entrylist}

%----------------------------------------------------------------------------------------
%	AWARDS SECTION
%----------------------------------------------------------------------------------------

\section{Awards and certifications}

\begin{entrylist}

%------------------------------------------------

\entry
    {2022}
    {Nanodegree - Site Reliability Engineer}
    {Udacity}

\entry
    {2021}
    {AWS Certified DevOps Engineer – Professional}
    {Amazon Web Services}

\entry
    {2020}
    {Google Cloud Platform Fundamentals: Core Infrastructure}
    {Coursera}

\entry
    {2019}
    {Honorable Mention}
    {Symposium on Circuits and Systems Design}
    {Award given on the 32nd SBCCI for the work titled "DNAr-Logic DNA circuit design library in R language for molecular computing".}

\entry
    {2015}
    {University Medal}
    {Federal University of Vi\c cosa}
    {The \textit{Presidente Bernardes} Medal is awarded to students with academic excellence.}

%------------------------------------------------

\end{entrylist}

%----------------------------------------------------------------------------------------
%	COMMUNICATION SKILLS SECTION
%----------------------------------------------------------------------------------------

\section{Communication skills}

\begin{entrylist}

%------------------------------------------------

\entry
{2017}
{Oral Presentation}
{Evostar Conference}
{Presented the research I conducted to obtain my Bachelor's degree. It was about the usage of Genetic Algorithms to optimize and solve a multi-component combinatorial problem.}

%------------------------------------------------

\end{entrylist}

%----------------------------------------------------------------------------------------
%	INTERESTS SECTION
%----------------------------------------------------------------------------------------

% \section{interests}

% \textbf{professional:} data analysis, company profiling, risk analysis, economics, web design, web app creation, software design, marketing \textbf{personal:} piano, chess, cooking, dancing, running

%----------------------------------------------------------------------------------------
%	PUBLICATIONS SECTION
%----------------------------------------------------------------------------------------

\section{Publications}

\printbibsection{article}{Articles in journals} % Print all articles from the bibliography

\printbibsection{inproceedings}{Conferences/proceedings}

% \printbibsection{book}{books} % Print all books from the bibliography

% \begin{refsection} % This is a custom heading for those references marked as "inproceedings" but not containing "keyword=france"
% \nocite{*}
% \printbibliography[sorting=chronological, type=inproceedings, title={international peer-reviewed conferences/proceedings}, notkeyword={france}, heading=bibheading]
% \end{refsection}

% \begin{refsection} % This is a custom heading for those references marked as "inproceedings" and containing "keyword=france"
% \nocite{*}
% \printbibliography[sorting=chronological, type=inproceedings, title={local peer-reviewed conferences/proceedings}, keyword={france}, heading=bibheading]
% \end{refsection}

% \printbibsection{misc}{other publications} % Print all miscellaneous entries from the bibliography

% \printbibsection{report}{research reports} % Print all research reports from the bibliography

%----------------------------------------------------------------------------------------

\clearpage
\restoregeometry
%}

\end{document}
