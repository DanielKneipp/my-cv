%%%%%%%%%%%%%%%%%%%%%%%%%%%%%%%%%%%%%%%%%
% cv-friggeri-x 1.0 (01/01/2016)
% XeLaTeX Template
%
% Based on:
% Friggeri Resume/CV
% Version 1.2 (3/5/15)
%
% Original author:
% Adrien Friggeri (adrien@friggeri.net)
% https://github.com/afriggeri/CV
% 
% Modified by:
% Nadorrano
% https://github.com/Nadorrano/cv-friggeri-x
%
% License:
% MIT (https://opensource.org/licenses/MIT)
% CC BY-NC-SA 3.0 (http://creativecommons.org/licenses/by-nc-sa/3.0/)
%
% Important notes:
% This template needs to be compiled with XeLaTeX and the bibliography,
% if used, needs to be compiled with biber rather than bibtex.
%
%%%%%%%%%%%%%%%%%%%%%%%%%%%%%%%%%%%%%%%%%

\documentclass[a4paper]{cv-friggeri-x}
% Add `a4paper` to set a4 paper size
% Add `nocolors` to remove colors from the document
% Add `lightheader` to change the dark background of the header to white

\usepackage{marvosym} % needed to print glyphs for email, cell phone etc.
\usepackage{soul} % used on the quote
\setstcolor{lightgray} 

\addbibresource{bibliography.bib} % Specify the bibliography file to include publications

\begin{document}

\header{Daniel}{ Kneipp}{Software Engineer}{``Do \st{what} the best you can, with what you have, where you are. -- Theodore Roosevelt''} % Your name and current job title/field

%----------------------------------------------------------------------------------------
%	SIDEBAR SECTION
%----------------------------------------------------------------------------------------

\begin{aside} % In the aside, each new line forces a line break
\section{Contact}
\pin \hfill 119 Gloriosa st.,
Belo Horizonte, MG 
305190-490, Brazil
~
% {\Large\textcolor{gray}{\Telefon}} \hfill +0 (000) 111 1111
{\Large\textcolor{gray}{\Mobilefone}} \hfill +55 (31) 9-9605 3234
{\Large\textcolor{gray}{\Letter}} \hfill \href{mailto:daniel.kneipp@outlook.com}{daniel.kneipp@\\outlook.com}
~
% \flogo \hfill \href{http://facebook.com/johnsmith}{fb://jsmith}
% \tlogo \hfill \href{http://twitter.com/johnsmith}{@jsmith}
\llogo \hfill \href{https://www.linkedin.com/in/daniel-kneipp/}{in://daniel-kneipp}
\githublogo \hfill \href{https://github.com/DanielKneipp}{github://DanielKneipp}
\gitlablogo \hfill \href{https://gitlab.com/DanielKneipp}{gitlab://DanielKneipp}
% \vklogo \hfill \href{http://vk.com/johnsmith}{vk://johnsmith}
% ~
% \href{http://www.smith.com}{http://www.smith.com}
\section{Languages}
Brazilian Portuguese \hspace{5mm}\null
\textit{[Mother tongue]}
English  \hspace{5mm}\null
\textit{[Professional working proficiency]}
\section{Programming}
C++, Python, R, JavaScript,
Matlab, Java, Bash
~
\section{Skills}
Machine Learning: \hspace{5mm}\null
\grade{4} 
Optimization:     \hspace{5mm}\null  
\grade{4}
Computer Vision:  \hspace{5mm}\null
\grade{3}
Text Mining:      \hspace{5mm}\null
\grade{2.5}
\end{aside}

%----------------------------------------------------------------------------------------
%	ABOUT ME SECTION
%----------------------------------------------------------------------------------------

\section{About me}
I am objective and focused on results. With entrepreneurship in my veins, I've spent many years studying and learning, from Convolutional Neural Networks to SWOT Analysis and Nash Equilibrium, trying to make a difference where I go. I know that it looks like a Miss Universe candidate speech asking for world peace (joke), but I really work hard to make some impact. Please, checkout my code repository (\href{https://github.com/DanielKneipp}{https://github.com/DanielKneipp}) to see some cool stuff.

%----------------------------------------------------------------------------------------
%	EDUCATION SECTION
%----------------------------------------------------------------------------------------

\section{Education}

\begin{entrylist}

\entry
    {2016--2018}
    {Master {\normalfont of Science}}
    {Federal University of Minas Gerais (UFMG)}
    {\emph{Comptuer Science}\\
    My research area is DNA Computing. The objective is to propose functional chemical circuits for classification tasks using Chemical Reaction Networks theory as a programming language and DNA strands as the hardware. I'm a NanoComp lab. Member (\href{http://www.nanocomp.dcc.ufmg.br/}{http://www.nanocomp.dcc.ufmg.br/}).}

\entry
    {2012--2015}
    {Bachelor {\normalfont of Science}}
    {Federal University of Vi\c cosa (UFV)}
    {\emph{Computer Science}\\
    I received Presidente Bernardes Medal for my academic excellence. Final Paper: A Genetic Algorithm for Multi-Component Optimization Problems: The Case of the Travelling Thief Problem.}


\entry
    {2010--2011}
    {Technician's {\normalfont Degree}}
    {SENAI School}
    {\emph{Informatics}\\
    I Studied the basics of Computer Architecture, Software Development and Network Infrastructure.}

\entry
    {}
    {\\Independent {\normalfont Courses}}
    {}
    {\begin{itemize}%
        \item Deep Learning -- \textit{Google} | \textit{Udacity}%
        \item Machine Learning -- \textit{Stanford University} | \textit{Coursera}%
    \end{itemize}}

%------------------------------------------------

\end{entrylist}

%----------------------------------------------------------------------------------------
%	WORK EXPERIENCE SECTION
%----------------------------------------------------------------------------------------

\section{Experience}

\subsection{Full time}

\begin{entrylist}

%------------------------------------------------

\entry
    {2017--Now}
    {Research and Development Analyst}
    {MOST Specialist Technologies}
    {Main activities:
    \begin{itemize}
        \item Clustering and analysis of textual medical records;
        \item Document classification based on its textual content;
        \item Modular deployment of solutions using Docker.
    \end{itemize}}

%------------------------------------------------

\end{entrylist}

\newpage

\subsection{Part time}

\begin{entrylist}

\entry
    {2016--2017}
    {Research Program}% RHAE SPECIAL PROGRAM
    {Invent Vision}
    {Deep Learning research for Computer Vision applications. Implementation of a set of tools to speedup the development (including synthetic dataset generation) and deployment of image classifiers. Application deployment in embedded systems (NVIDIA Jetson). Project name: Smart monitoring system by georeferenced images for railways applications.}

\entry
    {2015--2016}
    {Trainee}
    {Invent Vision}
    {Research and implantation of distributed computing systems (based on Hadoop and Spark), developing simple applications made to run across clusters.}

\entry
    {2013--2014}
    {Undergraduate Research}
    {Invent Vision}
    {Development of an efficient drowsiness detector based on face expressions (using face and eye tracking algorithms). Deployment made on x86 computers and ARM embedded systems. Project name: System for photometric inspection and automated adjustment of vehicle headlights. Project funded by CNPq (National Council for Scientific and Technological Development).}


%------------------------------------------------

\end{entrylist}

%----------------------------------------------------------------------------------------
%	AWARDS SECTION
%----------------------------------------------------------------------------------------

\section{Awards}

\begin{entrylist}

%------------------------------------------------

\entry
    {2015}
    {University Medal}
    {Federal University of Vi\c cosa}
    {The \textit{Presidente Bernardes} Medal is awarded to the students with academic excellence.}

%------------------------------------------------

\end{entrylist}

%----------------------------------------------------------------------------------------
%	COMMUNICATION SKILLS SECTION
%----------------------------------------------------------------------------------------

\section{Communication skills}

\begin{entrylist}

%------------------------------------------------

\entry
{2017}
{Oral Presentation}
{Evostar Conference}
{Presented the research I conducted to obtain my Bachelor's degree. It was about the usage of Genetic Algorithms to optimize and solve a multi-component combinatorial problem.}

%------------------------------------------------

\end{entrylist}

%----------------------------------------------------------------------------------------
%	INTERESTS SECTION
%----------------------------------------------------------------------------------------

% \section{interests}

% \textbf{professional:} data analysis, company profiling, risk analysis, economics, web design, web app creation, software design, marketing \textbf{personal:} piano, chess, cooking, dancing, running

%----------------------------------------------------------------------------------------
%	PUBLICATIONS SECTION
%----------------------------------------------------------------------------------------

\section{Publications}

\printbibsection{article}{Articles in journals} % Print all articles from the bibliography

\printbibsection{inproceedings}{International conferences/proceedings}

% \printbibsection{book}{books} % Print all books from the bibliography

% \begin{refsection} % This is a custom heading for those references marked as "inproceedings" but not containing "keyword=france"
% \nocite{*}
% \printbibliography[sorting=chronological, type=inproceedings, title={international peer-reviewed conferences/proceedings}, notkeyword={france}, heading=bibheading]
% \end{refsection}

% \begin{refsection} % This is a custom heading for those references marked as "inproceedings" and containing "keyword=france"
% \nocite{*}
% \printbibliography[sorting=chronological, type=inproceedings, title={local peer-reviewed conferences/proceedings}, keyword={france}, heading=bibheading]
% \end{refsection}

% \printbibsection{misc}{other publications} % Print all miscellaneous entries from the bibliography

% \printbibsection{report}{research reports} % Print all research reports from the bibliography

%----------------------------------------------------------------------------------------

\end{document}
