%%%%%%%%%%%%%%%%%%%%%%%%%%%%%%%%%%%%%%%%%
% cv-friggeri-x 1.0 (01/01/2016)
% XeLaTeX Template
%
% Based on:
% Friggeri Resume/CV
% Version 1.2 (3/5/15)
%
% Original author:
% Adrien Friggeri (adrien@friggeri.net)
% https://github.com/afriggeri/CV
% 
% Modified by:
% Nadorrano
% https://github.com/Nadorrano/cv-friggeri-x
%
% License:
% MIT (https://opensource.org/licenses/MIT)
% CC BY-NC-SA 3.0 (http://creativecommons.org/licenses/by-nc-sa/3.0/)
%
% Important notes:
% This template needs to be compiled with XeLaTeX and the bibliography,
% if used, needs to be compiled with biber rather than bibtex.
%
%%%%%%%%%%%%%%%%%%%%%%%%%%%%%%%%%%%%%%%%%

\documentclass[a4paper]{cv-friggeri-x}
% Add `a4paper` to set a4 paper size
% Add `nocolors` to remove colors from the document
% Add `lightheader` to change the dark background of the header to white

\usepackage{marvosym} % needed to print glyphs for email, cell phone etc.
\usepackage{soul} % used on the quote
\setstcolor{lightgray} 

\addbibresource{bibliography.bib} % Specify the bibliography file to include publications

\begin{document}

\header{Daniel}{ Kneipp}{Engenheiro de Software}{``Do \st{what} the best you can, with what you have, where you are. -- Theodore Roosevelt''} % Your name and current job title/field

%----------------------------------------------------------------------------------------
%	SIDEBAR SECTION
%----------------------------------------------------------------------------------------

\begin{aside} % In the aside, each new line forces a line break
\section{Contato}
\pin \hfill 119, rua Gloriosa,
Belo Horizonte, MG 
305190-490, Brasil
~
% {\Large\textcolor{gray}{\Telefon}} \hfill +0 (000) 111 1111
{\Large\textcolor{gray}{\Mobilefone}} \hfill +55 (31) 9-9605 3234
{\Large\textcolor{gray}{\Letter}} \hfill \href{mailto:daniel.kneipp@outlook.com}{daniel.kneipp@\\outlook.com}
~
% \flogo \hfill \href{http://facebook.com/johnsmith}{fb://jsmith}
% \tlogo \hfill \href{http://twitter.com/johnsmith}{@jsmith}
\llogo \hfill \href{https://www.linkedin.com/in/daniel-kneipp/}{in://daniel-kneipp}
\githublogo \hfill \href{https://github.com/DanielKneipp}{github://DanielKneipp}
\gitlablogo \hfill \href{https://gitlab.com/DanielKneipp}{gitlab://DanielKneipp}
% \vklogo \hfill \href{http://vk.com/johnsmith}{vk://johnsmith}
% ~
% \href{http://www.smith.com}{http://www.smith.com}
\section{Idiomas}
Português Brasileiro \hspace{5mm}\null
\textit{[língua materna]}
Inglês  \hspace{5mm}\null
\textit{[Competência profissional]}
\section{Programação}
C++, Python, R, JavaScript,
Matlab, Java, Bash
~
\section{Habilidades}
Aprendizado \hspace{5mm}\null
de Máquina: \hspace{5mm}\null
\grade{4} 
Otimização:     \hspace{5mm}\null  
\grade{4}
Visão Computacional:  \hspace{5mm}\null
\grade{3}
Mineração de Texto:      \hspace{5mm}\null
\grade{2.5}
\end{aside}

%----------------------------------------------------------------------------------------
%	ABOUT ME SECTION
%----------------------------------------------------------------------------------------

% \section{About me}
% I am objective and focused on results. With entrepreneurship in my veins, I've spent many years studying and learning, from Convolutional Neural Networks to SWOT Analysis and Nash Equilibrium, trying to make a difference where I go. I know that it looks like a Miss Universe candidate speech asking for world peace (joke), but I really work hard to make some impact. Please, checkout my code repository (\href{https://github.com/DanielKneipp}{https://github.com/DanielKneipp}) to see some cool stuff.


%----------------------------------------------------------------------------------------
%	WORK EXPERIENCE SECTION
%----------------------------------------------------------------------------------------

\section{Experiência}

\subsection{Tempo integral}

\begin{entrylist}

%------------------------------------------------

\entry
    {2017--Agora}
    {Analista de pesquisa e desenvolvimento}
    {MOST Specialist Technologies}
    {Principais tarefas:
    \begin{itemize}
        \item Agrupamento e análise de registro médicos textuais;
        \item Classificação de documentos baseado no conteúdo textual;
        \item Implantação modular de soluções usando Docker.
    \end{itemize}}

%------------------------------------------------

\end{entrylist}

% \newpage

\subsection{Meio período}

\begin{entrylist}

\entry
    {2016--2017}
    {Programa de Pesquisa}% RHAE SPECIAL PROGRAM
    {Invent Vision}
    {Pesquisa em \textit{Deeo Learning} para aplicações de visão computacional. Implementação de um conjunto de ferramentas para acelerar o desenvolvimento (incluindo geração de \textit{datasets} sintéticos) e implantação de classificadores de imagens. Implantação de aplicação em sistemas embarcados (NVIDIA Jetson). Nome do projeto: Sistema inteligente de monitoramento por imagens georreferenciadas para aplicação em ferrovias de carga.}

\entry
    {2015--2016}
    {Estagiário}
    {Invent Vision}
    {Pesquisa e implantação de sistemas de computação distribuído (baseados em Hadoop e Spark), desenvolvendo aplicações simples para serem executados em \textit{clusters}.}

\entry
    {2013--2014}
    {Iniciação Científica}
    {Invent Vision}
    {Desenvolvimento de um eficiente detector de fadiga baseado em expressões faciais (usando algoritmos de rastreamento de face e olhos). Implantação em computadores x86 e sistemas embarcados ARM. Nome do projeto: Sistema para inspeção fotométrica e regulagem automática de faróis de veículos automotores.}


%------------------------------------------------

\end{entrylist}

%----------------------------------------------------------------------------------------
%	EDUCATION SECTION
%----------------------------------------------------------------------------------------

\section{Educação}

\begin{entrylist}

\entry
    {2016--2018}
    {Mestrado}
    {Universidade Federal de Minas Gerais (UFMG)}
    {\emph{Ciência da Computação}\\
    Minha área de pesquisa é computação com DNA. O objetivo é propor circuitos químicos funcionais para tarefas de classificação usando a teoria das Redes de Reações Químicas como a linguagem de programação e fragmentos de DNA como o \textit{hardware}. Eu sou um membro do laboratório NanoComp (\href{http://www.nanocomp.dcc.ufmg.br/}{http://www.nanocomp.dcc.ufmg.br/}).}

\entry
    {2012--2015}
    {Bacharelado}
    {Universidade Federal de Vi\c cosa (UFV)}
    {\emph{Ciência da Computação}\\
    Recebi a medalha Presidente Bernardes pela minha excelência acadêmica. No meu trabalho de conclusão de curso desenvolvi um algorítimo baseado em um meta-heurística bio-inspirada para resolver um problema de otimização combinatório. Título: A Genetic Algorithm for Multi-Component Optimization Problems: The Case of the Travelling Thief Problem.}


\entry
    {2010--2011}
    {Técnico}
    {Escola SENAI}
    {\emph{Informática}\\
    Estudei o básico sobre arquitetura de computadores, desenvolvimento de software e infraestrutura de rede.}

% \entry
%     {}
%     {\\Independent {\normalfont Courses}}
%     {}
%     {\begin{itemize}%
%         \item Deep Learning -- \textit{Google} | \textit{Udacity}%
%         \item Machine Learning -- \textit{Stanford University} | \textit{Coursera}%
%     \end{itemize}}

%------------------------------------------------

\end{entrylist}

%----------------------------------------------------------------------------------------
%	AWARDS SECTION
%----------------------------------------------------------------------------------------

\section{Condecoreações}

\begin{entrylist}

%------------------------------------------------

\entry
    {2015}
    {Medalha Presidente Bernardes}
    {Universidade Federal de Vi\c cosa}
    {A medalha Presidente Bernardes é entregue a estudantes com excelência acadêmica na graduação.}

%------------------------------------------------

\end{entrylist}

%----------------------------------------------------------------------------------------
%	COMMUNICATION SKILLS SECTION
%----------------------------------------------------------------------------------------

\section{Habilidades de comunicação}

\begin{entrylist}

%------------------------------------------------

\entry
{2017}
{Apresentação oral}
{Conferência Evostar}
{Apresentei a pesquisa que conduzi para obter o título de Bacharel. O tema da pesquisa foi o uso de Algoritmos Genéticos para resolver um problema combinatório multicomponente.}

%------------------------------------------------

\end{entrylist}

%----------------------------------------------------------------------------------------
%	INTERESTS SECTION
%----------------------------------------------------------------------------------------

% \section{interests}

% \textbf{professional:} data analysis, company profiling, risk analysis, economics, web design, web app creation, software design, marketing \textbf{personal:} piano, chess, cooking, dancing, running

%----------------------------------------------------------------------------------------
%	PUBLICATIONS SECTION
%----------------------------------------------------------------------------------------

\section{Publicações}

\printbibsection{article}{Artigos em \textit{journals}} % Print all articles from the bibliography

\printbibsection{inproceedings}{Conferências internacionais}

% \printbibsection{book}{books} % Print all books from the bibliography

% \begin{refsection} % This is a custom heading for those references marked as "inproceedings" but not containing "keyword=france"
% \nocite{*}
% \printbibliography[sorting=chronological, type=inproceedings, title={international peer-reviewed conferences/proceedings}, notkeyword={france}, heading=bibheading]
% \end{refsection}

% \begin{refsection} % This is a custom heading for those references marked as "inproceedings" and containing "keyword=france"
% \nocite{*}
% \printbibliography[sorting=chronological, type=inproceedings, title={local peer-reviewed conferences/proceedings}, keyword={france}, heading=bibheading]
% \end{refsection}

% \printbibsection{misc}{other publications} % Print all miscellaneous entries from the bibliography

% \printbibsection{report}{research reports} % Print all research reports from the bibliography

%----------------------------------------------------------------------------------------

\end{document}
